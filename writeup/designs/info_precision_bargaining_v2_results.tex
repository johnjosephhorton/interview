\documentclass[11pt]{article}
\usepackage[margin=1in]{geometry}
\usepackage{graphicx, booktabs, hyperref, enumitem, xcolor, titlesec, parskip, amsmath}

\begin{document}

\begin{center}
{\Large \textbf{Experiment Results:\\The Information Precision Valley Under Tight Bargaining (v2)}}\\[0.5em]
{\large \today \quad $\cdot$ \quad Status: Post-analysis \quad $\cdot$ \quad Iteration of v1}
\end{center}

%% --- 1. RESEARCH QUESTION ---
\section*{Research Question}

Does tightening the zone of possible agreement (ZOPA) reveal an ``information valley'' in deal rates --- where range information about opponent valuations \emph{reduces} deal rates below both exact-information and no-information baselines?

The v1 experiment found 100\% deal rates across all conditions under generous ZOPAs (\$10--\$25), but discovered a significant monotonic price gradient ($F(2,27) = 7.99$, $p = 0.002$, $d = 1.54$). This v2 iteration shrinks the ZOPA to \$2--\$5 and reduces rounds from 6 to 3, hypothesizing that tight bargaining conditions would force impasses --- particularly in the range-information condition, where the provided range (\$35--\$50 for seller cost) spans values where no deal is feasible.

The core mechanism is \textbf{feasibility uncertainty}: when the buyer is told the seller's cost could be as high as \$50 but the buyer only values the item at \$42, the range includes values where no deal is profitable, potentially causing pessimistic anchoring and demands outside the true ZOPA. Triviality score: 21/25 (hypothesis memo v2).

%% --- 2. EXPERIMENTAL DESIGN ---
\section*{Experimental Design}

Three between-subjects conditions, each with a within-subject ZOPA manipulation via \texttt{buyer\_value} $\in \{42, 45\}$ (seller cost fixed at \$40, giving ZOPA = \$2 or \$5). Maximum 3 rounds (AI offers Rounds 1 and 3, human offers Round 2). Both agents are earnings maximizers with hard guardrails only.

\begin{table}[h!]
\centering
\begin{tabular}{@{}llp{6cm}@{}}
\toprule
Condition & Game folder & Information structure \\
\midrule
No Info & \texttt{bargain\_tight\_none} & Neither player knows opponent's valuation \\
Range Info & \texttt{bargain\_tight\_range} & Buyer told \$35--\$50; seller told \$38--\$55 \\
Exact Info & \texttt{bargain\_tight\_exact} & Both know both valuations \\
\bottomrule
\end{tabular}
\end{table}

\noindent All conditions share: seller\_cost = \$40, 3 rounds, alternating offers (AI first), price range \$0--\$100. The key design change from v1: ZOPA shrunk from \$10--\$25 to \$2--\$5, and rounds reduced from 6 to 3.

%% --- 3. DATA OVERVIEW ---
\section*{Data Overview}

\textbf{Sample:} 30 simulations total --- 10 per condition (5 at ZOPA = \$2, 5 at ZOPA = \$5). Factorial design with seed 1022.

\textbf{Checker results:} 27/30 nominal pass (90\%), 30/30 substantive pass (100\%). All 3 failures are the known checker false positive: deals at exactly \texttt{buyer\_value} (\$42) yield \$0.00 human earnings, and the checker hallucinates an arithmetic error (``incorrectly \$0.00 instead of \$0.00''). No substantive game issues.

\begin{table}[h!]
\centering
\begin{tabular}{@{}lrrrrr@{}}
\toprule
Condition & $N$ & Deal Rate & Mean Price & Mean AI Open & Mean Rounds \\
\midrule
No Info & 10 & 100\% & \$41.85 & \$49.50 & 2.6 \\
Range Info & 10 & 100\% & \$42.25 & \$45.00 & 2.5 \\
Exact Info & 10 & 100\% & \$41.80 & \$41.75 & 1.2 \\
\bottomrule
\end{tabular}
\caption{Summary statistics by condition. All 30 games reached deals.}
\end{table}

%% --- 4. RESULTS ---
\section*{Results}

\subsection*{Primary Outcome: Deal Rate}

The hypothesis predicted a valley: Range Info deal rate $<$ No Info $<$ Exact Info. \textbf{We observed 100\% deal rates across all conditions and all ZOPA levels.} The ceiling effect from v1 persists despite shrinking the ZOPA to \$2 and reducing rounds to 3. There is zero variance in the primary outcome --- no statistical test is possible.

All four directional predictions are \textbf{refuted}: the valley does not exist, not even under the tightest ZOPA (\$2). The feasibility uncertainty mechanism did not produce impasses.

\subsection*{Secondary Outcomes}

While the primary hypothesis failed, the secondary outcomes reveal how information precision shapes the \emph{process} of bargaining even when it does not affect the outcome.

\textbf{AI Opening Offers} (Figure~1, Panel A). Information precision dramatically affects anchoring behavior. One-way ANOVA: $F(2,27) = 8.75$, $p = 0.001$, $\eta^2 = 0.39$. The AI seller opens at the fair price when it knows the buyer's exact value (\$41.75), at a moderate anchor when given a range (\$45.00 --- exactly at the midpoint of the range \$38--\$55), and at an aggressive anchor when uninformed (\$49.50). Pairwise: exact vs.\ none $d = -1.50$ ($p = 0.008$); exact vs.\ range $d = -5.81$ ($p < 0.001$); range vs.\ none $d = 0.89$ ($p = 0.03$).

\textbf{Rounds to Deal} (Figure~1, Panel B). Information precision also affects convergence speed. Kruskal-Wallis $H = 16.1$, $p = 0.0003$, $\epsilon^2 = 0.52$. Exact-information pairs settle in a median of 1 round (the AI opens near fair price, the buyer accepts). No-info and range pairs require 2--3 rounds of negotiation. Exact vs.\ none and exact vs.\ range are highly significant ($p < 0.001$), but none vs.\ range do not differ.

\textbf{Deal Price} (Figure~1). Despite different opening offers, final deal prices do not significantly differ across conditions: $F(2,27) = 1.04$, $p = 0.37$. All conditions converge near the fair price (\$41--\$42.50 depending on ZOPA). The negotiation process corrects for aggressive anchoring within 3 rounds.

\textbf{Buyer Surplus Share.} Marginally significant: $F(2,27) = 3.36$, $p = 0.0497$. Buyers capture more surplus under exact info (47\%) than range info (30\%), with no info intermediate (45\%). Range vs.\ exact: $d = 1.04$, $p = 0.04$. This suggests range information creates an asymmetry that sellers exploit --- the buyer, uncertain whether a deal is even feasible, accepts seller-favorable prices.

\begin{figure}[h!]
\centering
\includegraphics[width=\textwidth]{../plots/results_info_precision_v2_price.pdf}
\caption{Deal price by information condition and ZOPA. Despite very different opening offers (AI opens at \$49.50 under no-info vs.\ \$41.75 under exact-info), final prices converge near the fair price in all conditions. Dashed lines show seller cost (\$40) and fair prices (\$41 and \$42.50).}
\end{figure}

\begin{figure}[h!]
\centering
\includegraphics[width=\textwidth]{../plots/results_info_precision_v2_dynamics.pdf}
\caption{(A)~Mean AI opening offer by condition. Information precision anchors the AI: exact info $\to$ fair price, range info $\to$ range midpoint, no info $\to$ aggressive anchor. (B)~Mean rounds to deal. Exact information produces near-instant agreement; no-info and range-info pairs negotiate for 2--3 rounds.}
\end{figure}

\begin{figure}[h!]
\centering
\includegraphics[width=\textwidth]{../plots/results_info_precision_v2_surplus.pdf}
\caption{Surplus distribution by condition and ZOPA. In the range-info condition, sellers capture a larger share of surplus (\$1.73/\$2 under tight ZOPA) than in the exact-info condition (\$1.10/\$2). Efficiency is 100\% in all conditions.}
\end{figure}

\subsection*{ZOPA Interaction}

ZOPA size does \emph{not} interact with information condition for any outcome. Two-way ANOVA on deal price: condition $\times$ ZOPA interaction $p = 0.68$. For AI opening offer: condition is the sole significant predictor ($p = 0.001$); ZOPA has no effect ($p = 0.58$), nor does the interaction ($p = 0.95$). The AI anchors identically regardless of whether the ZOPA is \$2 or \$5.

This directly refutes the moderation prediction: the hypothesized ``deeper valley under tight ZOPA'' does not exist because there is no valley at all.

%% --- 5. EVALUATION ---
\section*{Evaluation}

\subsection*{Did We Learn Something Interesting?}

\textbf{Partially --- we learned about negotiation process but not about deal breakdown.}

The primary hypothesis (information valley in deal rates) was comprehensively refuted: 100\% deal rates across all conditions, including under the tightest ZOPA (\$2) with only 3 rounds. The LLM agents are too rational and too cooperative to reach impasse. The ``feasibility uncertainty'' mechanism did not produce the predicted behavior --- even when the buyer was told the seller's cost could be as high as \$50 (above the buyer's own \$42 valuation), the simulated human still negotiated successfully.

However, the secondary findings are genuinely informative:

\begin{enumerate}[nosep]
    \item \textbf{Information anchoring is real and large} ($\eta^2 = 0.39$). The AI seller's opening offer is mechanically determined by its information: exact $\to$ fair price, range $\to$ range midpoint, none $\to$ aggressive anchor. This is the v2 analog of the v1 price gradient.
    \item \textbf{Negotiation corrects anchoring within 3 rounds.} Despite opening offers that differ by \$8 across conditions, final prices converge. The bargaining process is remarkably efficient.
    \item \textbf{Range info disadvantages buyers} (marginally significant). The buyer surplus share drops from 47\% to 30\% under range info, suggesting that feasibility uncertainty makes buyers more willing to accept seller-favorable terms even when deals are feasible.
\end{enumerate}

\textbf{Diagnosis: IMPLEMENTATION.} The ceiling effect is not a design problem (the ZOPA \emph{is} tight) but an implementation problem: gpt-5-nano agents reason about payoffs too well. They recognize that any positive earnings beat \$0 and accept deals even when the surplus is minimal. The feasibility uncertainty mechanism depends on \emph{behavioral} heuristics (anchoring, loss aversion, pessimism) that LLM agents do not exhibit. This is a fundamental limitation of LLM-as-subject experiments for testing behavioral hypotheses.

\subsection*{Limitations}

\begin{itemize}[nosep]
    \item \textbf{LLM rationality ceiling:} The agents compute that any deal $>$ \$0 beats impasse, eliminating the behavioral channel the hypothesis depends on. Human subjects, who exhibit loss aversion and anchoring, might produce the predicted valley.
    \item \textbf{Small N:} With only 10 per condition (5 per cell), we have limited power for secondary analyses. The buyer surplus share finding ($p = 0.0497$) is borderline.
    \item \textbf{Fixed ranges:} The ranges (\$35--\$50 and \$38--\$55) were chosen to span infeasible values, but the AI may not process ``infeasibility'' the way a human would.
    \item \textbf{Two iterations, same ceiling:} This is the second experiment (after v1) to find 100\% deal rates. The pattern is robust across ZOPAs from \$2 to \$25 and rounds from 3 to 6.
\end{itemize}

%% --- 6. NEXT EXPERIMENT ---
\section*{Next Experiment}

Two iterations have established that LLM agents do not produce impasse in bilateral bargaining, regardless of ZOPA size, round limits, or information structure. The ceiling effect is a property of the agents, not the game design. Three paths forward:

\textbf{Option A: PIVOT to a different outcome variable.} Instead of testing whether information affects \emph{deal rates} (which are always 100\%), test whether information affects the \emph{source of value} --- e.g., shift from bargaining over price to a game where information determines whether players can coordinate at all (e.g., a coordination game, common-pool resource, or signaling game where miscoordination is the failure mode rather than impasse).

\textbf{Option B: HUMAN-TEST the current design.} The tight-ZOPA bargaining game is well-designed for human subjects. The \$2 ZOPA with range info spanning infeasible values should produce genuine feasibility uncertainty in human participants. Run the v2 games with real humans via \texttt{interview chat}.

\textbf{Option C: PIVOT to information \emph{source} effects.} The v1 results showed a price gradient driven by anchoring. Rather than manipulating precision (none/range/exact), manipulate the \emph{source} of information: does the same numerical information carry different weight depending on whether it comes from a credible vs.\ non-credible source? This shifts the mechanism from behavioral heuristics (which LLMs lack) to information processing (which LLMs can model).

\textbf{Recommended: Option C (PIVOT).} The anchoring result (opening offers scale with information) is the most robust finding across both v1 and v2. A source-credibility manipulation directly tests whether LLM agents \emph{discount} information based on its provenance --- a cognitive mechanism that LLMs may actually exhibit, unlike loss aversion or feasibility pessimism.

\textbf{Next step:} Run \texttt{/hypothesize} with a new direction: ``Does information source credibility moderate anchoring effects in bargaining? When told opponent valuations come from an unreliable source, do agents anchor less aggressively?''

\subsection*{Prior Experiment Context (Machine-Readable)}
\begin{verbatim}
prior_hypothesis: Under tight bargaining (ZOPA <=$5, <=3 rounds), range info
  reduces deal rates below exact and no-info baselines via feasibility uncertainty
verdict: NO_NULL
diagnosis: IMPLEMENTATION
key_finding: 100% deal rates across all conditions even with $2 ZOPA and 3 rounds;
  LLM agents too rational for impasse. Secondary: significant anchoring effect on
  AI opening offers (eta-sq=0.39) and marginal buyer surplus disadvantage under
  range info.
key_statistic: deal_rate = 100% (30/30); AI opening offer F(2,27)=8.75, p=.001
dag_variables: X=info_precision, M=feasibility_uncertainty, Y=deal_rate, Z=zopa_size
testable_implications_results: valley_in_deal_rates=REFUTED,
  no_info_resilience=CONFIRMED (100% deals), exact_info_efficiency=CONFIRMED
  (fastest deals), zopa_moderation=REFUTED (no interaction)
next_archetype: PIVOT
proposed_changes: Abandon deal-rate hypothesis for LLM agents. Pivot to
  information source credibility as a manipulation that LLMs can process.
  Alternatively, test the current design with human subjects.
next_hypothesis_sketch: Information source credibility moderates anchoring in
  bargaining. Valuations from unreliable sources produce weaker anchoring and
  more aggressive counteroffers than identical valuations from reliable sources.
\end{verbatim}

\end{document}
