\documentclass[11pt]{article}
\usepackage[margin=1in]{geometry}
\usepackage{amsmath, graphicx, booktabs, hyperref, enumitem, xcolor, titlesec, parskip}

\definecolor{accentblue}{HTML}{2563EB}
\definecolor{mutedgray}{HTML}{6B7280}

\hypersetup{colorlinks=true, linkcolor=accentblue, urlcolor=accentblue}

\title{\textbf{Hypothesis Memo:} \\ Same Information, Different Source --- \\ Why Cheap Talk Outperforms Verified Disclosure in Bargaining}
\author{Pre-design}
\date{February 2026}

\begin{document}
\maketitle
\thispagestyle{empty}

\noindent\textit{Status: Pre-design --- ready for \texttt{/design-experiment}}

\section{Research Question}

\textbf{Does the \emph{source} of partial information about the bargaining zone --- game-verified disclosure vs.\ the opponent's unverified claim (cheap talk) --- have opposite effects on deal rates, even when the information content is identical?}

In two-player bargaining, a zone of possible agreement (ZOPA) may exist but remain hidden from one or both players. A natural intervention is to provide partial information about the opponent's valuation to help players find mutually acceptable terms. But information can arrive through different channels: it can be \emph{verified} (disclosed by the game rules, known to be true) or \emph{unverified} (claimed by the opponent, who may be lying). Standard economic theory treats these symmetrically --- rational agents should update on verified information and discount cheap talk by the sender's incentive to deceive. Behaviorally, however, the source may matter enormously.

We hypothesize a \emph{source-channel reversal}: game-verified disclosure of the opponent's approximate valuation \emph{decreases} deal rates relative to no information, because it activates exploitative anchoring (the informed player pushes toward the opponent's limit). Conversely, the same information conveyed as an unverified opponent claim \emph{increases} deal rates, because the act of voluntary disclosure triggers a reciprocal social process --- the opponent ``chose to share,'' creating perceived obligation and cooperative framing. If confirmed, this implies that the behavioral channel (reciprocity vs.\ exploitation) dominates the informational channel (content accuracy), and that \emph{less credible} information can produce \emph{better} outcomes.

\section{Interestingness Argument}

\subsection*{Triviality Scorecard}

\begin{table}[h!]
\centering
\small
\begin{tabular}{@{}lcp{9cm}@{}}
\toprule
\textbf{Dimension} & \textbf{Score} & \textbf{Reasoning} \\
\midrule
Prediction surprise & 4/5 & The dominant intuition is that verified information is more useful than cheap talk. The prediction that cheap talk \emph{outperforms} verified disclosure --- same content, opposite effects --- is surprising. Experts would split: game theorists expect verified info to dominate; behavioral economists might see the reciprocity channel but would not confidently predict a full reversal. \\[0.5em]
Literature gap & 4/5 & Cheap talk in bargaining is well-studied (Farrell \& Gibbons 1989; Valley et al.\ 2002), and information disclosure in auctions/bargaining is a major literature. But the \emph{direct comparison} of same-content information from different sources (verified game rule vs.\ unverified opponent claim) in bilateral bargaining lacks experimental evidence. The source-credibility literature in persuasion does not extend cleanly to bargaining. \\[0.5em]
Mechanism specificity & 4/5 & Two competing mechanisms are identified and the design distinguishes them. Game disclosure activates \emph{exploitative anchoring} (negative channel): the player with verified information knows where to push. Opponent claims activate \emph{reciprocal trust} (positive channel): voluntary sharing creates cooperative obligations. Holding information content constant isolates the source channel. \\[0.5em]
Boundary conditions & 4/5 & The hypothesis is a moderated effect, not a main effect: information helps or hurts depending on \emph{source}. Additionally, we predict this reversal is strongest when the ZOPA is narrow (small surplus makes exploitation more damaging and reciprocity more valuable). \\[0.5em]
Testability in games & 5/5 & The outcome is binary (deal/no-deal), directly observable in transcripts. The treatment is clean: three between-subjects conditions (no info, game-disclosed, opponent-claimed). First offers and concession patterns are observable for mediation analysis. \\
\midrule
\textbf{Total} & \textbf{21/25} & Score $\geq 18$: proceed as-is. \\
\bottomrule
\end{tabular}
\end{table}

\noindent No sharpening required. The hypothesis already features a source-channel reversal (same content, opposite effects by source), two competing mechanisms cleanly separated by design, and a secondary moderator (ZOPA size). The core surprise --- that less credible information produces better outcomes --- is a strong, testable claim.

\section{Causal Model}

\begin{figure}[h!]
    \centering
    \includegraphics[width=\textwidth]{plots/dag.pdf}
    \caption{Pearlean causal DAG for the information-source bargaining hypothesis. The treatment (blue) is the source of partial information about the opponent's valuation: no information, game-verified disclosure, or opponent's unverified claim. Two competing mediators (gray) represent the behavioral channels activated by each source. Game disclosure activates exploitative anchoring ($M_1$, red negative arrow): the informed player anchors offers near the opponent's reservation price. Opponent claims activate reciprocal trust ($M_2$, green positive arrow): voluntary sharing creates cooperative obligation. The outcome (green) is whether the pair reaches a deal. ZOPA size and information content (dashed border) are held constant across conditions. The key prediction: $M_1$ dominates under game disclosure (reducing deals) while $M_2$ dominates under opponent claims (increasing deals), producing opposite effects from identical information content.}
    \label{fig:dag}
\end{figure}

\subsection*{Variable Definitions}

\begin{table}[h!]
\centering
\small
\begin{tabular}{@{}llp{4.0cm}p{4.2cm}@{}}
\toprule
\textbf{Variable} & \textbf{Type} & \textbf{Operationalization} & \textbf{Game Measurement} \\
\midrule
$X$ & Treatment & Information source: (1) no info about opponent's valuation, (2) game discloses opponent's approximate valuation (verified), (3) opponent claims their own approximate valuation in a chat phase (unverified) & Game variant: three configs differing only in how/whether info about the opponent's reservation price reaches the player \\[0.3em]
$M_1$ & Mediator & Exploitative anchoring: first offer's proximity to the opponent's reservation price & Anchoring index: $\frac{|offer_1 - V_{opponent}|}{ZOPA}$; lower = more exploitative \\[0.3em]
$M_2$ & Mediator & Reciprocal trust: cooperative framing in offers and concession speed & Concession rate across rounds; cooperative language in cheap-talk transcripts (where available) \\[0.3em]
$Y$ & Outcome & Deal reached: binary indicator of whether players agree on a price within ZOPA before time expires & Transcript: explicit accept/reject of a final price \\[0.3em]
$Z$ & Control & ZOPA size (\$20 in all conditions); information content (same approximate valuation range disclosed regardless of source) & Identical valuation draws and ZOPA across all three conditions \\
\bottomrule
\end{tabular}
\end{table}

\subsection*{Testable Implications}

\begin{enumerate}[label=\arabic*., leftmargin=*, itemsep=4pt]
    \item \textbf{Source-channel reversal:} Deal rate in the opponent-claimed condition $>$ no-info baseline $>$ game-disclosed condition. Same information content, opposite directional effects depending on source.
    \item \textbf{Anchoring mechanism:} In the game-disclosed condition, first offers are significantly closer to the opponent's reservation price (higher anchoring index) than in the no-info or opponent-claimed conditions.
    \item \textbf{Reciprocity mechanism:} In the opponent-claimed condition, concession rates are faster (players converge more quickly) than in the no-info condition, consistent with reciprocal trust driving cooperation.
    \item \textbf{Mediation:} Controlling for the anchoring index ($M_1$) should attenuate the negative effect of game disclosure on deal rates. Controlling for concession speed ($M_2$) should attenuate the positive effect of opponent claims.
\end{enumerate}

\subsection*{Identification Strategy}

\begin{itemize}[leftmargin=1.5em, itemsep=3pt]
    \item \textbf{Randomized:} Information source (three-level treatment) is randomized via game variant assignment. Each session is one of: no-info, game-disclosed, opponent-claimed.
    \item \textbf{Held constant:} ZOPA size (buyer valuation $-$ seller cost $= \$20$), number of bargaining rounds, AI strategy (identical \texttt{player.md}), payoff structure, turn structure. Crucially, the \emph{information content} is identical across the two informed conditions --- only the source differs.
    \item \textbf{Key design constraint:} In the opponent-claimed condition, the AI must be scripted to ``voluntarily'' share approximately the same valuation information that the game discloses in the game-disclosed condition. This ensures content equivalence. The AI's claim is approximately truthful (matching the verified disclosure) so that the comparison isolates source, not accuracy.
    \item \textbf{Confounds ruled out:} Random assignment eliminates selection. Identical ZOPA rules out difficulty differences. Identical AI strategy rules out strategic adaptation. Content equivalence across informed conditions isolates the source channel.
    \item \textbf{Limitations:} (1)~The opponent-claimed condition requires the AI to make a ``voluntary'' disclosure, which is scripted rather than truly voluntary --- this may not fully capture the social dynamics of real cheap talk. (2)~We cannot directly observe beliefs (does the player \emph{trust} the AI's claim?) --- only actions. (3)~The AI opponent is an LLM, not a human; the reciprocity mechanism may be weaker against a known AI. (4)~Three conditions require larger sample sizes than a two-condition design to detect the interaction.
\end{itemize}

\section{Next Steps}

This hypothesis is ready for \texttt{/design-experiment} to map to a concrete game design. The design phase will need to:

\begin{itemize}[leftmargin=1.5em, itemsep=2pt]
    \item Define the three game variants (no-info, game-disclosed, opponent-claimed) with identical payoff structures
    \item Script the AI's ``voluntary disclosure'' in the opponent-claimed condition to match the game's disclosure in the verified condition (content equivalence)
    \item Decide whether to include ZOPA-size variation as a secondary treatment (narrow vs.\ wide ZOPA) to test the predicted moderator
    \item Set the number of bargaining rounds (balancing data richness against learning effects)
    \item Choose the approximate valuation signal: e.g., ``the opponent's valuation is between \$X and \$Y'' (same range in both informed conditions)
\end{itemize}

\end{document}
