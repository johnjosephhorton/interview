\documentclass[11pt]{article}
\usepackage[margin=1in]{geometry}
\usepackage{graphicx, booktabs, hyperref, enumitem, xcolor, titlesec, parskip}

\definecolor{scoreblue}{HTML}{2166ac}

\begin{document}

\begin{center}
{\Large \textbf{Hypothesis Memo: Cheap Talk $\times$ Information Asymmetry in Bargaining}}\\[0.5em]
{\large \today \quad $\cdot$ \quad Status: Pre-design}
\end{center}

%% ─────────────────────────────────────────────
\section*{Research Question}

\textbf{Does cheap talk increase deal rates under one-sided private information but \emph{decrease} deal rates under two-sided private information, in bargaining games where a mutually beneficial deal exists?}

Standard theory predicts that asymmetric information causes bargaining failures even when gains from trade exist (Myerson \& Satterthwaite, 1983). A natural prescription is to allow communication: let players talk before making offers. But cheap talk---costless, non-binding messages---creates a dual channel. Players can use it to \emph{signal} their position credibly (facilitating deals) or to \emph{misrepresent} strategically (destroying trust). Which channel dominates likely depends on the information structure. Under one-sided asymmetry, only one player has private information and incentive to bluff; the uninformed party can partially update on signals. Under two-sided asymmetry, \emph{both} players have incentive to misrepresent, creating a fog of strategic communication that may leave both parties worse off than silence.

%% ─────────────────────────────────────────────
\section*{Interestingness Argument}

\subsection*{Triviality Scorecard}

\begin{table}[h!]
\centering
\small
\begin{tabular}{@{}lcp{9cm}@{}}
\toprule
\textbf{Dimension} & \textbf{Score} & \textbf{Reasoning} \\
\midrule
Prediction surprise & 4/5 & The crossover interaction---cheap talk \emph{helps} under one-sided but \emph{hurts} under two-sided asymmetry---is counter-intuitive. Most practitioners assume communication always helps. Experts would split on the direction under two-sided asymmetry. \\[0.5em]
Literature gap & 4/5 & Cheap talk in bargaining is studied (Farrell \& Gibbons, 1989; Valley et al., 2002), and asymmetric-information bargaining is textbook. But the specific $2 \times 2$ interaction---cheap talk $\times$ one-sided vs.\ two-sided asymmetry---lacks direct experimental evidence, especially with LLM agents. \\[0.5em]
Mechanism specificity & 4/5 & Two competing mechanisms identified: \emph{credible signaling} (positive channel) vs.\ \emph{strategic misrepresentation} (negative channel). The design distinguishes them by varying information structure, which shifts the balance between channels. \\[0.5em]
Boundary conditions & 5/5 & This \emph{is} an interaction effect: cheap talk $\times$ information structure. The predicted direction \emph{reverses} across conditions---not a universal main effect. \\[0.5em]
Testability in games & 5/5 & Deal/no-deal is a clean binary outcome directly observable in transcripts. The treatment is a simple $2 \times 2$ between-subjects factorial. \\
\midrule
\textbf{Total} & \textbf{22/25} & Score $\geq 18$: proceed as-is. \\
\bottomrule
\end{tabular}
\end{table}

\noindent No sharpening required. The hypothesis already features a crossover interaction effect, competing mechanisms, and a clean experimental design. The prediction that cheap talk \emph{backfires} under two-sided asymmetry is the key novel claim.

%% ─────────────────────────────────────────────
\section*{Causal Model}

\begin{figure}[h!]
\centering
\includegraphics[width=\textwidth]{plots/dag.pdf}
\caption{Pearlean causal DAG for the cheap talk $\times$ information asymmetry hypothesis. Two treatment variables (blue)---information structure ($X_1$: one-sided vs.\ two-sided private information) and cheap talk ($X_2$: allowed vs.\ not)---affect deal rate ($Y$, green) through two competing mediators (gray): strategic misrepresentation ($M_1$, negative pathway) and credible signaling ($M_2$, positive pathway). Under two-sided asymmetry, both players misrepresent, making $M_1$ dominant; under one-sided asymmetry, signaling ($M_2$) dominates. ZOPA size ($Z$, dashed border) is held constant across conditions.}
\end{figure}

\subsection*{Variable Definitions}

\begin{table}[h!]
\centering
\small
\begin{tabular}{@{}llp{4.2cm}p{4.2cm}@{}}
\toprule
\textbf{Variable} & \textbf{Type} & \textbf{Operationalization} & \textbf{Game Implementation} \\
\midrule
$X_1$ & Treatment & Information structure: one-sided (only seller knows own cost) vs.\ two-sided (both have private valuations) & Game variant: one version reveals buyer's valuation to both; the other keeps both valuations private \\[0.3em]
$X_2$ & Treatment & Communication: cheap talk phase allowed vs.\ no communication & Game variant: one version includes a free-text chat phase before offers; the other skips to offers \\[0.3em]
$M_1$ & Mediator & Strategic misrepresentation: degree to which players make false claims about valuations during chat & Measured from chat transcripts: count of claims inconsistent with true valuations \\[0.3em]
$M_2$ & Mediator & Credible signaling: degree to which players truthfully reveal preference information & Measured from chat transcripts: count of truthful valuation-relevant statements \\[0.3em]
$Y$ & Outcome & Deal rate: fraction of games where players reach a mutually agreed price within the ZOPA & Binary per game: did both players accept a price? Aggregate: proportion across sessions \\[0.3em]
$Z$ & Control & Zone of possible agreement (ZOPA) size & Held constant: buyer valuation $-$ seller cost $= \$20$ in all conditions \\
\bottomrule
\end{tabular}
\end{table}

\subsection*{Testable Implications}

\begin{enumerate}[label=\arabic*.]
\item \textbf{Main interaction:} Deal rate in the (cheap talk $+$ one-sided asymmetry) condition $>$ (no talk $+$ one-sided) condition, \emph{but} deal rate in the (cheap talk $+$ two-sided asymmetry) condition $<$ (no talk $+$ two-sided) condition.
\item \textbf{Mechanism test:} In the two-sided/cheap-talk cell, the rate of strategic misrepresentation ($M_1$) should be significantly higher than in the one-sided/cheap-talk cell.
\item \textbf{Mediation:} Controlling for misrepresentation intensity ($M_1$) should attenuate the negative effect of cheap talk under two-sided asymmetry.
\item \textbf{Boundary:} If ZOPA is very large (easy deals), the negative effect of cheap talk under two-sided asymmetry should shrink---misrepresentation matters less when there is ample surplus.
\end{enumerate}

\subsection*{Identification Strategy}

\begin{itemize}[leftmargin=1.5em]
\item \textbf{Randomization:} Participants are randomly assigned to one of four conditions in a $2 \times 2$ between-subjects design: \{one-sided, two-sided\} $\times$ \{cheap talk, no talk\}.
\item \textbf{Held constant:} Payoff structure (ZOPA $= \$20$), number of bargaining rounds, AI opponent strategy (identical across conditions), and game framing.
\item \textbf{Confounds ruled out:} Random assignment eliminates selection effects. Identical ZOPA rules out difficulty differences. Same AI strategy rules out strategic adaptation confounds.
\item \textbf{Limitations:} (1)~Misrepresentation ($M_1$) and signaling ($M_2$) are measured from transcripts via content coding, which introduces measurement noise. (2)~The design cannot distinguish whether cheap talk backfires because of \emph{active} misrepresentation or \emph{passive} confusion (players talking past each other). (3)~LLM agents may not misrepresent in the same way humans do.
\end{itemize}

%% ─────────────────────────────────────────────
\section*{Next Steps}

This hypothesis is ready for \texttt{/design-experiment} to map the $2 \times 2$ factorial to concrete game specifications (config, manager prompt, player prompt). The key design challenge is operationalizing ``one-sided'' vs.\ ``two-sided'' private information in a way that feels natural to players and produces enough variation in deal rates to detect the interaction effect in $\sim$50--100 sessions per cell.

\end{document}
