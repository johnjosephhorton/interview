\documentclass[11pt]{article}
\usepackage[margin=1in]{geometry}
\usepackage{graphicx, booktabs, hyperref, enumitem, xcolor, titlesec, parskip}

\definecolor{darkblue}{HTML}{1E40AF}
\definecolor{darkgreen}{HTML}{047857}

\begin{document}

\begin{center}
{\Large \textbf{Hypothesis Memo: The Information Precision Valley}}\\[0.5em]
{\large \today \quad $\cdot$ \quad Status: Pre-design}
\end{center}

%% --- 1. RESEARCH QUESTION ---
\section*{Research Question}

Does partial information about opponent valuations create an ``information valley'' in bargaining --- reducing deal rates below both full-information and no-information baselines?

Standard bargaining theory predicts that more information monotonically improves efficiency: information asymmetries are the primary source of failed negotiations (Myerson \& Satterthwaite, 1983), so revealing more should help. But this prediction conflates two distinct functions of information: (1)~\emph{focal point coordination} (exact values provide a natural split-the-difference anchor) and (2)~\emph{aspiration calibration} (any information about the opponent shapes what you think you can get). We hypothesize that partial information --- knowing a range but not exact values --- activates the second function without the first, creating a paradoxical zone where \emph{more knowledge produces worse outcomes}.

The core intuition: when you know the opponent's valuation is somewhere between \$50 and \$80, you anchor to the extreme that favors you (a buyer assumes the seller's cost is near \$50; the seller assumes the buyer's value is near \$80). Both sides form inflated aspirations. But neither has a precise focal point to coordinate on, so they negotiate aggressively from incompatible positions. Under no information, risk aversion drives exploratory compromise. Under exact information, the known ZOPA provides a natural coordination point. Range information gives the worst of both worlds: confident enough to be aggressive, but too uncertain to coordinate.

This matters because most real-world negotiations involve partial information --- parties know \emph{something} about each other's position, but rarely exact values. If partial information is genuinely the worst regime, it reframes information disclosure policies: vague transparency (``the seller paid between \$200K and \$300K'') may be worse than either full disclosure or no disclosure at all.

%% --- 2. INTERESTINGNESS ARGUMENT ---
\section*{Interestingness Argument}

\begin{table}[h!]
\centering
\begin{tabular}{@{}lcp{9.5cm}@{}}
\toprule
\textbf{Dimension} & \textbf{Score} & \textbf{Reasoning} \\
\midrule
Prediction surprise & 4 & Most researchers predict monotonic improvement with more information. The non-monotonic ``valley'' --- partial info is \emph{worst} --- contradicts the standard intuition. Some information economists might predict this, but most experimentalists would not. \\[0.5em]
Literature gap & 4 & Information precision as a continuous variable (none $\to$ range $\to$ exact) has theoretical treatments in mechanism design, but direct experimental comparison of all three precision levels in the same bargaining game is rare. Prior work in this project tested info \emph{source} (verified vs.\ claims) and info \emph{asymmetry} (one-sided vs.\ two-sided), but not \emph{precision}. \\[0.5em]
Mechanism specificity & 4 & Two competing mechanisms --- aspiration inflation (drives failure under range info) vs.\ focal point coordination (drives success under exact info) --- and the 3-condition design isolates each. Comparing opening offers across conditions tests aspiration inflation directly. \\[0.5em]
Boundary conditions & 4 & ZOPA tightness moderates: the valley should be deepest with tight ZOPA (inflated aspirations eat a small surplus) and shallowest with wide ZOPA (room for both aspirations). This is a clean, testable interaction. \\[0.5em]
Testability in games & 5 & Deal rate is binary and cleanly observable. Opening offer aggressiveness (distance from the fair split) directly measures aspiration inflation. Deal prices under exact info test focal point clustering. \\
\midrule
\textbf{Total} & \textbf{21/25} & Proceed as-is (strong hypothesis). \\
\bottomrule
\end{tabular}
\caption*{Triviality scorecard. Score $\geq$ 18: proceed without sharpening.}
\end{table}

No sharpening needed. The hypothesis already features a non-monotonic prediction, two identified mechanisms, a moderating interaction (ZOPA size), and cleanly observable outcomes.

%% --- 3. CAUSAL MODEL ---
\section*{Causal Model}

\begin{figure}[h!]
\centering
\includegraphics[width=\textwidth]{plots/dag.pdf}
\caption{Causal DAG for the Information Precision Valley hypothesis. Information Precision (treatment, 3 levels) affects Deal Rate through two competing mediating pathways: \emph{aspiration inflation} (activated by range information, reduces deals) and \emph{focal point availability} (activated by exact information, increases deals). ZOPA Size moderates by amplifying aspiration inflation when surplus is small.}
\end{figure}

\subsection*{Variable Definitions}

\begin{table}[h!]
\centering
\begin{tabular}{@{}llp{4.8cm}p{4.8cm}@{}}
\toprule
\textbf{Variable} & \textbf{Type} & \textbf{Operationalization} & \textbf{Game implementation} \\
\midrule
$X$ & Treatment & Information precision: None / Range / Exact & 3 game conditions varying what players are told about each other's valuation \\[0.3em]
$M_1$ & Mediator & Aspiration inflation: opening offer aggressiveness & Distance of first offer from the ZOPA midpoint \\[0.3em]
$M_2$ & Mediator & Focal point availability: convergence target & Clustering of deal prices around ZOPA midpoint (exact info) vs.\ dispersion (range/none) \\[0.3em]
$Y$ & Outcome & Deal rate; rounds to deal; surplus efficiency & Binary deal/no-deal; round count; (deal price $-$ seller cost) / ZOPA \\[0.3em]
$Z$ & Moderator & ZOPA size: tight (\$10) vs.\ wide (\$25) & Controlled by buyer\_value and seller\_cost parameters \\
\bottomrule
\end{tabular}
\end{table}

\subsection*{Testable Implications}

\begin{enumerate}[leftmargin=*, itemsep=0.3em]
    \item \textbf{The valley:} Deal rate under Range info $<$ deal rate under No info $<$ deal rate under Exact info. The non-monotonic ordering is the central prediction.
    \item \textbf{Aspiration inflation:} Opening offers in the Range condition are more aggressive (further from the fair split) than in No info or Exact info conditions.
    \item \textbf{Focal point clustering:} Deal prices in the Exact info condition cluster near the ZOPA midpoint; prices in Range and No info are more dispersed.
    \item \textbf{ZOPA moderation:} The deal rate gap between Range and Exact conditions is larger with tight ZOPA (\$10) than wide ZOPA (\$25). With wide ZOPA, all conditions should achieve high deal rates (ceiling effect suppresses the valley).
\end{enumerate}

\subsection*{Identification Strategy}

\begin{itemize}[leftmargin=*, itemsep=0.3em]
    \item \textbf{Randomize:} Information precision (3 levels) $\times$ ZOPA size (2 levels) = 6 cells, between-subjects
    \item \textbf{Hold constant:} Number of rounds (6), no cheap talk, both agents are earnings maximizers with identical goal prompts, same underlying valuation draws within ZOPA-size conditions
    \item \textbf{Ruled out:} AI strategy confounds (same earnings-maximization goal across all conditions), ZOPA variation within cells (controlled parameterization), cheap talk confounds (excluded)
    \item \textbf{Unidentified:} Whether ``aspiration inflation'' is the true cognitive mechanism vs.\ some other process --- we observe aggressive offers, not beliefs. Also, AI agents may not exhibit human-like anchoring biases, limiting external validity.
\end{itemize}

%% --- 4. NEXT STEPS ---
\section*{Next Steps}

This hypothesis is ready for \texttt{/design-experiment} to map it to a concrete game design. The design should produce 6 game conditions (3 precision levels $\times$ 2 ZOPA sizes), with a tight ZOPA condition calibrated to produce real impasse risk (deal rates between 40--80\%, not ceiling 100\%). The prior \texttt{info\_source\_bargaining} experiment showed that a \$30 ZOPA produced 100\% deal rates --- the tight condition here should use $\sim$\$10 to create variance.

Run: \texttt{/design-experiment info\_precision\_bargaining}

\end{document}
